\documentclass[12pt]{article}
\usepackage{amsmath}
\usepackage{mathtools}
\usepackage{xcolor}
\usepackage{amssymb}

\begin{document}

\begin{center}
\normalsize{CPSC 2101: Programming Methodology Lab} \\
\vspace{.2cm}
\large{A Brief Introduction to Matrix Operations} \\
\end{center}

\section{What is a Matrix?}
A matrix is an array representation of data grouped in rows and columns. We can write a matrix as follows:
\[
M =
	\begin{bmatrix} 
		1 & 2 & 3 \\
		4 & 5 & 6 \\
		7 & 8 & 9
	\end{bmatrix}
\]
To refer to an element $a$ in matrix $M$ we denote it by row $i$, and column $j$, where $0 \leqslant i \leqslant 2$ and $0 \leqslant j \leqslant 2$. For example, $a_{0,0} = 1$ and $a_{2,2} = 9$. Next, we will discuss various operations that can be performed on a matrix.

\section{Matrix Operations}

There are several matrix operations that we will look at. Specifically we will discuss matrix addition, subtraction, multiply and transpose.

\subsection{Matrix Addition}

Matrix addition is a matrix x matrix operation meaning that it requires two matrices.
\begin{center}
$M = \begin{bmatrix} 5 & 2 & 1 \\ 4 & 2 & 4 \\ 1 & 3 & 1 \end{bmatrix}
  N = \begin{bmatrix} 3 & 1 & 6 \\ 5 & 2 & 2 \\ 4 & 3 & 3 \end{bmatrix}
$
\end{center}
\begin{center}
$M + N =
 \begin{bmatrix} 5 & 2 & 1 \\ 4 & 2 & 4 \\ 1 & 3 & 1 \end{bmatrix}  +
  \begin{bmatrix} 3 & 1 & 6 \\  5 & 2 & 2 \\ 4 & 3 & 3 \end{bmatrix} = 
  \begin{bmatrix} 5 + 3 & 2 + 1 & 1 + 6 \\ 4 + 5 & 2 + 2 & 4 + 2 \\ 1 + 4 & 3 + 3 & 1 + 3 \end{bmatrix} =
  \begin{bmatrix} 8 & 3 & 7 \\ 9 & 4 & 6 \\ 5 & 6 & 4 \end{bmatrix}
$
\end{center}
Note the two matrices above. $M + N$ is the sum from an element $a$ in $M$ and $b$ in $N$. So, for each $i, j$ in the result matrix is $a_{i,j} + b_{i,j}$ or:

\begin{center}
$M + N =
 \begin{bmatrix} a_{0, 0} + b_{0, 0} & a_{0, 1} + b_{0, 1} & a_{0, 2} + b_{0, 2} \\
 		         a_{1, 0} + b_{1, 0} & a_{1, 1} + b_{1, 1} & a_{1, 2} + b_{1, 2} \\
		         a_{2, 0} + b_{2, 0} & a_{2, 1} + b_{2, 1} & a_{2, 2} + b_{2, 2}
 \end{bmatrix}
$
\end{center}

\subsection{Matrix Subtraction}

Matrix subtraction is very similar to matrix addition. We simply perform a subtraction operation on each element instead of addition.
\begin{center}
$M = \begin{bmatrix} 5 & 2 & 1 \\ 4 & 2 & 4 \\ 1 & 3 & 1 \end{bmatrix}
  N = \begin{bmatrix} 3 & 1 & 6 \\ 5 & 2 & 2 \\ 4 & 3 & 3 \end{bmatrix}
$
\end{center}
\begin{center}
$M - N =
 \begin{bmatrix} 5 & 2 & 1 \\ 4 & 2 & 4 \\ 1 & 3 & 1 \end{bmatrix}  -
  \begin{bmatrix} 3 & 1 & 6 \\  5 & 2 & 2 \\ 4 & 3 & 3 \end{bmatrix} = 
  \begin{bmatrix} 5 - 3 & 2 - 1 & 1 - 6 \\ 4 - 5 & 2 - 2 & 4 - 2 \\ 1 - 4 & 3 - 3 & 1 - 3 \end{bmatrix} =
  \begin{bmatrix} 2 & 1 & -5 \\ -1 & 0 & 2 \\ -3 & 0 & -2 \end{bmatrix}
$
\end{center}
The corresponding matrix operations that are performed can be seen below:
\begin{center}
$M + N =
 \begin{bmatrix} a_{0, 0} - b_{0, 0} & a_{0, 1} - b_{0, 1} & a_{0, 2} - b_{0, 2} \\
 		         a_{1, 0} - b_{1, 0} & a_{1, 1} - b_{1, 1} & a_{1, 2} - b_{1, 2} \\
		         a_{2, 0} - b_{2, 0} & a_{2, 1} - b_{2, 1} & a_{2, 2} - b_{2, 2}
 \end{bmatrix}
$
\end{center}

\subsection{Matrix Multiplication}

Matrix multiplication is different from matrix addition and subtraction in that we do not simply multiply each element in a matrix $M$ by each element in matrix $N$. We can write matrix multiplication for a single element as:

$$(M N)_{i, j} = \sum_{k = 1}^{n} M_{i, k}N_{k, j}$$

If we look at the examples we have been using we get the resulting matrix:
 \begin{center}
$M = \begin{bmatrix} 5 & 2 & 1 \\ 4 & 2 & 4 \\ 1 & 3 & 1 \end{bmatrix}
  N = \begin{bmatrix} 3 & 1 & 6 \\ 5 & 2 & 2 \\ 4 & 3 & 3 \end{bmatrix}
$
\end{center}
\begin{center}
$M * N =
 \begin{bmatrix} 5 & 2 & 1 \\ 4 & 2 & 4 \\ 1 & 3 & 1 \end{bmatrix}  *
  \begin{bmatrix} 3 & 1 & 6 \\  5 & 2 & 2 \\ 4 & 3 & 3 \end{bmatrix} = 
  \begin{bmatrix} (5*3) + (2*5) + (1*4) & (5*1) + (2*2) + (1*3) & (5*6) + (2*2) + (1*3) \\ 
                           (4*3) + (2*5) + (4*4) & (4*1) + (2*2) + (4*3) & (4*6) + (2*2) + (4*3) \\ 
                           (1*3) + (3*5) + (1*4) & (1*1) + (3*2) + (1*3) & (1*6) + (3*2) + (1*3) 
  \end{bmatrix} =
  \begin{bmatrix} 29 & 12 & 37 \\ 38 & 20 & 40 \\ 22 & 10 & 15 \end{bmatrix}
$
\end{center}
We can write the matrix as follows:
\begin{center}
$
\begin{bsmallmatrix} 
		         (a_{0, 0} * b_{0, 0}) + (a_{0, 1} * b_{1, 0}) + (a_{0, 2} * b_{2, 0}) & 
                          (a_{0, 0} * b_{0, 1}) + (a_{0, 1} * b_{1, 1}) + (a_{0, 2} * b_{2, 1}) & 
                          (a_{0, 0} * b_{0, 2}) + (a_{0, 1} * b_{1, 2}) + (a_{0, 2} * b_{2, 2}) \\
                          (a_{1, 0} * b_{0, 0}) + (a_{1, 1} * b_{1, 0}) + (a_{1, 2} * b_{2, 0}) & 
                          (a_{1, 0} * b_{0, 1}) + (a_{1, 1} * b_{1, 1}) + (a_{1, 2} * b_{2, 1}) & 
                          (a_{1, 0} * b_{0, 2}) + (a_{1, 1} * b_{1, 2}) + (a_{1, 2} * b_{2, 2}) \\
                          (a_{2, 0} * b_{0, 0}) + (a_{2, 1} * b_{1, 0}) + (a_{2, 2} * b_{2, 0}) & 
                          (a_{2, 0} * b_{0, 1}) + (a_{2, 1} * b_{1, 1}) + (a_{2, 2} * b_{2, 1}) & 
                          (a_{2, 0} * b_{0, 2}) + (a_{2, 1} * b_{1, 2}) + (a_{2, 2} * b_{2, 2}) \\
 \end{bsmallmatrix}
$
\end{center}

\subsection{Matrix Scale}

The final operation we will discuss is matrix scaling. In matrix scaling, we multiply a matrix by a single scalar value.
\begin{center}
$M = \begin{bmatrix} 5 & 2 & 1 \\ 4 & 2 & 4 \\ 1 & 3 & 1 \end{bmatrix}
  s = 2
$
\end{center}
\begin{center}
$M * s =
 \begin{bmatrix} 5 & 2 & 1 \\ 4 & 2 & 4 \\ 1 & 3 & 1 \end{bmatrix}  *
  2 = 
  \begin{bmatrix} 5 * 2 & 2 * 2 & 1 * 2 \\ 
                           4 * 2 & 2 * 2 & 4 * 2 \\ 
                           1 * 2 & 3 * 2 & 1 * 2
  \end{bmatrix} =
  \begin{bmatrix} 10 & 4 & 2 \\ 8 & 4 & 8 \\ 2 & 6 & 2 \end{bmatrix}
$
\end{center}
We write this matrix as follows:
\begin{center}
$M * s=
 \begin{bmatrix} a_{0, 0} * s & a_{0, 1} * s & a_{0, 2} * s \\
 		         a_{1, 0} * s & a_{1, 1} * s & a_{1, 2} * s \\
		         a_{2, 0} * s & a_{2, 1} * s & a_{2, 2} * s
 \end{bmatrix}
$
\end{center}

\subsection{Matrix Transposition}

Matrix transposition is a unary operation that requires only a single matrix. For a matrix $M$, a matrix transposition denoted as $M^T$, replaces an arbitrary element $a_{i, j}$ with element $a_{j, i}$. For example let's use the following matrix:
\begin{center}
$M = \begin{bmatrix} 1 & 2 & 3 \\ 4 & 5 & 6 \\ 7 & 8 & 9 \end{bmatrix}$
\end{center}
If we apply the transpose operation on $M$, we get:
\begin{center}
$M^T = \begin{bmatrix} 1 & 4 & 7 \\ 2 & 5 & 8 \\ 3 & 6 & 9 \end{bmatrix}$
\end{center}
So, a general matrix transpose operation would look like this:
\begin{center}
$M = 
 \begin{bmatrix} a_{0, 0} & a_{0, 1} & a_{0, 2} \\
 		         a_{1, 0} & a_{1, 1} & a_{1, 2} \\
		         a_{2, 0} & a_{2, 1} & a_{2, 2}
 \end{bmatrix},
 M^T = 
  \begin{bmatrix} a_{0, 0} & a_{1, 0} & a_{2, 0} \\
 		          a_{0, 1} & a_{1, 1} & a_{2, 1} \\
		          a_{0, 2} & a_{1, 2} & a_{2, 2}
 \end{bmatrix}
$
\end{center}

\subsection{Matrix Symmetry}
Matrix symmetry is a property of a square matrix. If a matrix is symmetric, this means that a matrix $M = M^T$.
For example:
\begin{center}
$M = \begin{bmatrix} 1 & 2 & 3 \\ 4 & 5 & 6 \\ 7 & 8 & 9 \end{bmatrix}$
\end{center}
is not symmetric. If we take $M^T$, it does not equal $M$.
\begin{center}
$M' = \begin{bmatrix} \color{red} 1 & 4 & 7 \\ 4 & \color{red}5 & 6 \\ 7 & 6 & \color{red}9 \end{bmatrix}$
\end{center}
However, matrix $M'$ is symmetric. Note that the elements on either side of the diagonal for matrix $M'$ are the same on both sides.
\end{document}
